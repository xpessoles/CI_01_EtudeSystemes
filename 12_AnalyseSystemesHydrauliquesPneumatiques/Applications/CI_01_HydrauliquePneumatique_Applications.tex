\documentclass[10pt]{article}
\input{style/coursHeadings}
\input{style/programHeadings}
\input{style/macros_SII}
\input{style/macros_Titres}
\input{style/macros_Frames}

%Si le boolen xp est vrai : compilation pour xabi
%Sinon compilation Damien
\newboolean{xp}
\setboolean{xp}{true}

\newboolean{prof}
\setboolean{prof}{true}

\usepackage[%
    pdftitle={Systèmes hydrauliques et pneumatiques},
    pdfauthor={Xavier Pessoles},
    colorlinks=true,
    linkcolor=blue,
    citecolor=magenta]{hyperref}


\def\discipline{Sciences Industrielles de l'Ingénieur}
\def\xxtitre{\ifthenelse{\boolean{xp}}{
CI 1 : Analyse des systèmes pluritechniques et multiphysiques -- Initiation à l'Ingénierie Système}{}}

\def\xxsoustitre{\ifthenelse{\boolean{xp}}{
Chapitre 12 -- Analyse des systèmes hydrauliques et pneumatiques}{
Partie  -- }}

\def\xxauteur{\ifthenelse{\boolean{xp}}{
Xavier \textsc{Pessoles}\\
2013 -- 2014}{}}

\def\xxpied{\ifthenelse{\boolean{xp}}{
CI 1 : Analyse des systèmes pluritechniques et multiphysiques\\
Ch. 12 : Systèmes hydrauliques et pneumatiques -- Cours}{
\xxtitre}}

\def\xxcathegorie{\ifthenelse{\boolean{xp}}{
2013 -- 2014 \\
Xavier \textsc{Pessoles}}{
Informatique - Cours}}

%---------------------------------------------------------------------------


\begin{document}

\ifthenelse{\boolean{xp}}{\input{style/enteteXP}}{\input{style/enteteDI}}

\begin{center}
\textit{\Large{Exercice d'application}}
\end{center}


\subsection*{Schéma hydraulique du pilote automatique de voilier}
\vspace{.5cm}
\begin{center}
\includegraphics[width=.4\textwidth]{images/GroupeHydraulique}
\hfill
\includegraphics[width=.4\textwidth]{images/GroupeHydraulique}
\end{center}

\subparagraph{}
\textit{Donner la désignation de chacun des composants hydrauliques.}

\subparagraph{}
\textit{Analyser ce schéma dans les cas d’utilisation  suivants :
\begin{itemize}
\item le vérin sort à droite;
\item le vérin sort à gauche;
\item l’effort sur le vérin devient trop important;
\item le vérin est immobile mais doit rester en place sous les efforts;
\item le vérin doit se déplacer librement car le safran est sous commande manuelle (le volant du bateau).
\end{itemize}}

\subparagraph{}
\textit{Proposer un schéma cinématique de la pompe (on précise qu'il s'agit d'une pompe à pistions radiaux).}
\end{document}


