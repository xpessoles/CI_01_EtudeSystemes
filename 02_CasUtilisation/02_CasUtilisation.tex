\documentclass[11pt,oneside]{article}
\input{coursHeadings}
\usepackage[raccourcis]{FAST}
\usepackage[%
    pdftitle={IS -- SysML -- Cas d'utilisation},
    pdfauthor={Xavier Pessoles},
    colorlinks=true,
    linkcolor=blue,
    citecolor=magenta]{hyperref}

\usepackage{pifont}


% \makeatletter \let\ps@plain\ps@empty \makeatother
%% DEBUT DU DOCUMENT
%% =================
\sloppy
\hyphenpenalty 10000

\newcommand{\Pointilles}[1][3]{%
\multido{}{#1}{\makebox[\linewidth]{\dotfill}\\[\parskip]
}}


\colorlet{shadecolor}{orange!15}

\newtheorem{theorem}{Theorem}


\begin{document}


\newboolean{prof}
\setboolean{prof}{true}
%------------- En tetes et Pieds de Pages ------------
\pagestyle{fancy}
\renewcommand{\headrulewidth}{0pt}

\fancyhead{}
\fancyhead[L]{%
\noindent\noindent\begin{minipage}[c]{2.6cm}
%Lycée Rouvière PTSI
\includegraphics[width=2cm]{png/logo_ptsi.png}%
\end{minipage}
}


\fancyhead[C]{\rule{12cm}{.5pt}}

\fancyhead[R]{%
\noindent\begin{minipage}[c]{3cm}
\begin{flushright}
\footnotesize{\textit{\textsf{Sciences Industrielles\\ de l'Ingénieur}}}%
\end{flushright}
\end{minipage}
}

\renewcommand{\footrulewidth}{0.2pt}

\fancyfoot[C]{\footnotesize{\bfseries \thepage}}
\fancyfoot[L]{\footnotesize{2013 -- 2014} \\ X. \textsc{Pessoles}}
\ifthenelse{\boolean{prof}}{%
\fancyfoot[R]{\footnotesize{
CI 1 : IS -- Cours \\
Ch 2 SysML -- UC -- P}}
}{%
\fancyfoot[R]{\footnotesize{Cours -- CI 6 : PPM}}
}


\begin{center}
 \huge\textsc{CI 1 -- IS}

 \large\textsc{Étude des systèmes pluritechniques et multiphysiques -- Initiation à l'Ingénierie Système}
\end{center}

\begin{center}
 \LARGE\textsc{Chapitre 2 -- SysML -- Diagramme des cas d'utilisation}
\end{center}

%\begin{flushright}
%\textit{D'après documents de Jean-Pierre Pupier}
%\end{flushright}

\vspace{.5cm}





%\begin{minipage}[c]{.23\linewidth}
%\begin{center}
%\includegraphics[height=2.5cm]{png/tour_bois}
%
%\textit{Tour à bois \cite{tab}}
%\end{center}
%\end{minipage} \hfill
%\begin{minipage}[c]{.23\linewidth}
%\begin{center}
%\includegraphics[height=2.5cm]{png/cazeneuve}
%
%\textit{Tour conventionnel \cite{cazeneuve}}
%\end{center}
%\end{minipage} \hfill
%\begin{minipage}[c]{.23\linewidth}
%\begin{center}
%\includegraphics[height=2.5cm]{png/tour_mazak}
%
%\textit{Tour à commande numérique \cite{mazak}}
%\end{center}
%\end{minipage}\hfill
%\begin{minipage}[c]{.23\linewidth}
%\begin{center}
%\includegraphics[height=2.5cm]{png/plaquettes}
%
%\textit{Plaquettes de tournage \cite{plaquettes}}
%\end{center}
%\end{minipage}

\vspace{.5cm}

\begin{center}
%\includegraphics[height=7cm]{png/cycleV}
\end{center}

%\begin{center}
%\includegraphics[width=.9\textwidth]{png/cyclev.png}

%\textit{Cycle de conception d'un produit}
%\end{center}

%\begin{prob}
%\textsc{Problématique :}
%\begin{itemize}
%\item %Quelles sont les conditions fonctionnelles permettant le fonctionnement du système ?
%\item %Quelle est la chaîne de côte unidirectionnelle correspondant à une condition donnée ?
%\end{itemize}
%\end{prob}



\begin{savoir}
\textsc{Savoirs :}
\begin{itemize}
\item A-C2.1	: Frontière d'étude, fonction globale et performance, cas d’utilisation, acteurs (humain ou systèmes connectés), interactions fonctionnelles, relations entre cas d’utilisation	
\item A-C2.2	: Diagramme des cas d’utilisation de sysML	
\end{itemize}
\end{savoir}
 

Le diagramme des cas d'utilisation (\textit{Use case diagram} -- \textit{uc}) est adapté pour montrer les interactions entre les acteurs et le système étudié.

\begin{center}
\includegraphics[width=0.9\textwidth]{png/diagrammes}
\end{center}


\setlength{\parskip}{0ex plus 0.2ex minus 0ex}
 \renewcommand{\contentsname}{}
 \renewcommand{\baselinestretch}{1}

\tableofcontents

 \renewcommand{\baselinestretch}{1.2}
\setlength{\parskip}{2ex plus 0.5ex minus 0.2ex}

% \vspace{1cm}
\textit{Ce document évolue. Merci de signaler toutes erreurs ou coquilles.}

\section{Présentation}


\begin{defi}
\textbf{Système \cite{roques}}

\begin{minipage}[c]{.75\textwidth}
Un système est un ensemble de composants interreliés qui interagissent les uns avec les autres d’une manière organisée pour accomplir une finalité commune.% (NASA 1995).

Un système est un ensemble intégré d’éléments qui accomplissent un objectif défini (INCOSE \footnotemark[1]  2004).

Le système est délimité par une frontière représentée par un cadre. Cette frontière doit être clairement définie.
\end{minipage}\hfill
\begin{minipage}[c]{.2\textwidth}
\begin{center}
\includegraphics[width=.9\textwidth]{png/frontiere}
\end{center}
\end{minipage}
\end{defi}
\footnotetext[1]{International Council on Systems Engineering}

\begin{defi}
\textbf{Acteur}

\begin{minipage}[c]{.8\textwidth}
Un acteur est une entité qui interagit avec le système. On entend par interaction un échange de matière, d'énergie ou d'information. 

Un acteur humain est représenté par un bonhomme de fer.
\end{minipage}\hfill
\begin{minipage}[c]{.15\textwidth}
\begin{center}
\includegraphics[width=.5\textwidth]{png/utilisateur}
\end{center}
\end{minipage}
\end{defi}


\begin{exemple}
Dans la majorité des systèmes, l'acteur est l'utilisateur principal : dans le cas de la voiture, l'acteur principal est le conducteur. 

Dans le cadre d'un réseau informatique, l'acteur peut être un ordinateur client. 
\end{exemple}



\begin{defi}
\textbf{Cas d'utilisation}

Également appelé fonctionnalité ou service rendu, le cas d'utilisation est énoncé par l'utilisateur principal qui demande ce qu'il attend du système. 


Un cas d'utilisation possède : 
\begin{itemize}
\item un élément déclencheur;
\item une succession d'étape;
\item un élément de fin qui témoigne que le service a été rendu.
\end{itemize}

\vspace{.25cm}

\begin{minipage}[c]{.7\textwidth}
Les cas d'utilisation sont représentés par des ovales. Ils sont reliés par un trait à l'acteur principal. Ce trait est appelé lien d'association.
\end{minipage}\hfill
\begin{minipage}[c]{.25\textwidth}
\begin{center}
\includegraphics[width=.9\textwidth]{png/casutilisation}
\end{center}
\end{minipage}
\end{defi}

\begin{defi}
Diagramme des cas d'utilisation -- \textit{Use case diagram} -- \textit{uc}

Le diagramme des cas d'utilisation regroupe donc les différents cas d'utilisation d'un système.

\end{defi}


\begin{exemple}
\textit{Conduite d'un véhicule personnel}

\begin{center}
\includegraphics[width=.5\textwidth]{png/uc_utiliser}
\end{center}

\end{exemple}

\begin{rem}
De façon générale, on s'attachera à avoir un nombre restreint de cas d'utilisation pour un système. 

Dans le cas des système du laboratoire, on ne dépasse généralement pas les 2 à 3 cas. 
\end{rem}

\section{Pour aller plus loin}

\subsection{Acteur généralisé}
\begin{defi}
\textbf{Généralisation -- Spécialisation}

Dans un système, un même type d'acteur peut se spécialiser suivant l'utilisation qu'il a du système. Ainsi un acteur spécialisé aura les mêmes propriété que l'acteur généralisé. On dit que l'utilisateur spécialisé \textbf{hérite} de la description de son parent. Il peut en outre avoir des propriétés supplémentaires. 

\begin{minipage}[c]{.6\linewidth}
La relation de généralisation est noté par une flèche dont la pointe est vide. La flèche désigne le parent. 
\end{minipage} \hfill
\begin{minipage}[c]{.3\linewidth}
\begin{center}
\includegraphics[width=.7\textwidth]{png/gene}
\end{center}
\end{minipage} 

\end{defi}


\begin{exemple}
\textit{Utilisation d'un véhicule personnel}

Dans le cas de l'utilisation d'un véhicule personnel, un occupant du véhicule peut être passager ou conducteur. Chacun des utilisateurs est un humain, mais n'a pas les mêmes attentes vis-à-vis du véhicule. 

\begin{center}
\includegraphics[width=.5\textwidth]{png/uc_utiliser}
\end{center}

\end{exemple}

\subsection{Les relations entre cas }
\begin{defi}
\textbf{Inclusion -- \textit{include}}

Lorsqu'un "sous-cas" est inclus dans un cas "prinicpal", cela signifie que le "sous-cas" est obligatoirement exécuté lors de la réalisation du cas "principal".

La relation d'inclusion peut permettre de décomposer un cas complexe en cas élémentaires. 

\begin{minipage}[c]{.7\textwidth}
La relation d'inclusion est symbolisée par une flèche en traits interrompus portant le \textbf{stéréotype} \textit{include}. La flèche pointe le cas principal.
\end{minipage}\hfill
\begin{minipage}[c]{.25\textwidth}
\begin{center}
\includegraphics[width=.6\textwidth]{png/include}
\end{center}
\end{minipage}




\end{defi}

\begin{defi}
\textbf{Extension -- \textit{extend}}

Un cas d'utilisation principal peut être étendu lorsque une seconde fonctionnalité existe sur le système mais qu'elle n'est pas systématiquement utilisée.

\begin{minipage}[c]{.7\textwidth}
La relation d'extension est symbolisée par une flèche en traits interrompus portant le \textbf{stéréotype} \textit{extend}. La flèche pointe le cas principal.
\end{minipage}\hfill
\begin{minipage}[c]{.25\textwidth}
\begin{center}
\includegraphics[width=.6\textwidth]{png/extend}
\end{center}
\end{minipage}

\end{defi}

\begin{defi}
\textbf{Généralisation -- spécialisation}

Pour montrer qu'un cas d'utilisation est un cas particulier d'un autre il est possible d'utiliser une flèche simple. 

\end{defi}

\begin{exemple}
\textit{Balance Halo}

La balance Halo est une balance ménagère électronique disposant des fonctionnalités classiques d'une balance (pesée, tare...). Elle permet aussi de faire des conversions de masse en volume. 
\begin{minipage}[c]{.3\textwidth}
\begin{center}
\includegraphics[width=.6\textwidth]{png/balanceHalo_2}
\end{center}
\end{minipage}\hfill
\begin{minipage}[c]{.65\textwidth}
\begin{center}
\includegraphics[width=.7\textwidth]{png/balanceHalo}
\end{center}
\end{minipage}

L'extension montre la possibilité de demander à la balance une conversion en volume. 

L'inclusion montre que la balance dispose de la fonction tarer. Cette fonctionnalité étant utilisée à chaque démarrage de la balance, elle n'est donc pas optionnelle.
\end{exemple}

\subsection{Description des cas d'utilisation}

Suivant la complexité des systèmes il peut être difficile de comprendre la (ou les) fonctionnalité(s) attendues à partir du diagramme des cas d'utilisation. Il est alors possible d'utiliser un descriptif. Le mode d'écriture de ce descriptif n'est pas normalisé. 

Il peut prendre la forme suivante :
\begin{center}
\begin{tabular}{|p{4cm}|p{10cm}|}
\hline 
Nom du cas d’utilisation & Commentaires \\ \hline
Exigences associées & Indiquent les exigences auxquelles ce cas d’utilisation répond partiellement ou totalement. \\ \hline
Conditions de succès & Indiquent les conditions qui mènent à un service effectivement rendu. \\ \hline
Conditions d’échec & Indiquent les conditions qui mènent à un service non rendu. \\ \hline
Déclencheur & L’événement déclenché par un acteur qui provoque l’exécution du cas d’utilisation. \\ \hline
Flux principal & Description de chaque étape importante de l’exécution normale du cas d’utilisation. \\ \hline
etc &  \\ \hline
\end{tabular}
\end{center}

\begin{exemple}
\textit{Balance Halo}

\begin{center}
\begin{tabular}{|p{4cm}|p{10cm}|}
\hline 
Nom du cas d’utilisation & Peser des aliments
 \\ \hline
Exigences associées & \\ \hline
Conditions de succès & Le poids total ne doit pas dépasser 2 kg. \\ \hline
Conditions d’échec & Le poids dépasse 2 kg. \\ \hline
Déclencheur & L’utilisateur appuie sur un bouton.\\ \hline
Flux principal & 
\begin{itemize}
\item L’utilisateur appuie sur le bouton de mise en marche;
\item la balance se tare et affiche « 0 g » une fois le tarage terminé;
\item il dépose sur le plateau ce qu’il souhaite;
\item l’affichage se met à jour en temps réel;
\item la balance s’éteint au bout d’un certain temps d’inactivité.
\end{itemize}
\\ \hline
Extensions &
  \begin{itemize}
\item Après dépose d’un aliment l’utilisateur peut tarer la balance par appuie sur
un bouton;
\item l’utilisateur peut demander à convertir le poids en volume d’eau.
\end{itemize}\\ \hline
\end{tabular}
\end{center}
\end{exemple}




\begin{thebibliography}{2}
\bibitem{roques}{Pascal Roques, SysML par l'exemple -- Un langage de modélisation pour systèmes complexes. Éditions Eyrolles, 2009.}
\bibitem{debout}{Pierre Debout, SII -- Analyse Externe des systèmes.}
\bibitem{martin}{Beaudoin Martin, Formation SysML.}
\bibitem{martin2}{Beaudoin Martin, Construction	du	modèle	SysML	de	la	balance		
HALO	de	chez	Terraillon.}
\bibitem{martin3}{Beaudoin Martin, Diagrammes SysML -- L'essentiel en STI2D.}
\bibitem{sanford}{Sanford Friedenthal, Alan Moore, Rick Steiner, A Practical Guide to SysML-- The Systems Modeling Language. Elsevier, 2008.}
\end{thebibliography}

\end{document}

\section{Exemple}


Permet d'avoir un point de vue utilisateur / acteur
acteur = humain ou autre système interacteur

fonctionnalités visibles de l'extérieur (mais pas toutes les fonctionnalités)


Acteur = bonhomme batonbaton

cadre = frontière du système



Acteur : 
un humain ayant un rôle qui va interagir avec le système
un système
tout ce qui est en INTERaction avec le système

acteur primaire plutot a gauche
acteur secondaire plutot a droite

interaction = "communication" et pas d'interaction purement physique


Frontière du système : représenté par un cadre
le cadre symbolise la frontièer du système

cette frontière doit être bien comprise et définie. 
elle englobe ce qui n'est pas présent avant l'installation du système.
Le reste représente souvent un acteur



Cas d'utilisation dans un ovale

Il est énoncé par un acteur PRIMAIRE

L'acteur dit " moi je veux"
Il énonce le résultat qu'il attend
Mission que l'on cherche à atteindre à travers le système

Cas d'utilisation possède un élément déclencheur
Succession d'étape
Le service est rendu

Mettre des éléments qui interagissent réellement avec le système (pas le soleil, le vent ...)




(Attention : maintenance = exigence)


Descritpion des scénaros
decrit le scenario nominal
les scenarios alternarifs
les scénarios d'echecs
mais aussi les préconditions (l "état du systeme pour que la cas d'utilisation puisse démarrer)
les postconditions (ce qui a changé dans l'état du système a la fin du cas d'utilisation)


Attention
ne pas descendre trop bas en terme de détail (cibler les fonctionnalités premieres)
le cas d'utilisation ne doit donc pas se réduire systèmatiquement a une seule sequence
encore moins a une simple action


Pour aller plus loin
relation entre cas d'utilisation
inclusion - stereotype include
extension - extend
généralisation / spécialisation (fleche blanche) pas de stereotype
