\documentclass[10pt]{article}
\input{style/coursHeadings}
\input{style/programHeadings}
\input{style/macros_SII}
\input{style/macros_Titres}
\input{style/macros_Frames}

%Si le boolen xp est vrai : compilation pour xabi
%Sinon compilation Damien
\newboolean{xp}
\setboolean{xp}{true}

\newboolean{prof}
\setboolean{prof}{true}

\newboolean{td}
\setboolean{td}{true}

\usepackage[%
    pdftitle={},
    pdfauthor={Xavier Pessoles},
    colorlinks=true,
    linkcolor=blue,
    citecolor=magenta]{hyperref}


\def\discipline{Sciences Industrielles de l'Ingénieur}
\def\xxtitre{\ifthenelse{\boolean{xp}}{
CI 1 : Étude des systèmes pluritechniques et multiphysiques -- Initiation à l'Ingéniérie Système}{}}

\def\xxsoustitre{\ifthenelse{\boolean{xp}}{
Chapitre 1 -- Introduction à l'Ingénérie Systèmes}{
Partie  -- }}

\def\xxauteur{\ifthenelse{\boolean{xp}}{
Xavier \textsc{Pessoles}}{}}

\def\xxpied{\ifthenelse{\boolean{xp}}{
CI 1 : IS \\
Ch. 1 : Introduction à l'IS -- Cours TD}{
\xxtitre}}

\def\xxcathegorie{\ifthenelse{\boolean{xp}}{
2013 -- 2014 \\
Xavier \textsc{Pessoles}}{
Informatique - Cours}}





%---------------------------------------------------------------------------


\begin{document}

\ifthenelse{\boolean{xp}}{\input{style/enteteXP}}{\input{style/enteteDI}}



 \renewcommand{\baselinestretch}{1.2}
\setlength{\parskip}{2ex plus 0.5ex minus 0.2ex}


\begin{comp}
\noindent \textbf{Analyser :} 
\begin{itemize}
\item \textit{A1} : identifier le besoin et définir les exigences du système;
\item \textit{A2} : définir les frontières de l'analyse;
\item \textit{A3} : conduire l'analyse (\textit{A3--C3}).
\end{itemize}

\end{comp}

\section*{Machine de rééducation Sys-Reeduc}

\begin{flushright}
\textit{D'après concours CCP -- MP -- 2013.}
\end{flushright}

%\ifprof
%\else

\begin{minipage}[c]{.7\linewidth}
Fruit d'un projet régional entre le CReSTIC\footnote{Centre de Recherche en Sciences et Technologies de l'Information et de la Communication.} de Reims et le CRITT-MDTS\footnote{Centre Régional d'Innovation et de Transfert de Technologie.} de Charleville-Mézières, le Sys-Reeduc est un système permettant d'aider à la rééducation des membres inférieurs. 
\begin{obj} 
Le but de ce TD est d'analyser et de comprendre le fonctionnement du Sys-Reeduc.
\end{obj}

\end{minipage} \hfill
\begin{minipage}[c]{.27\linewidth}
\begin{center}
\includegraphics[width=\textwidth]{images/Sys_Reeduc_01}
\end{center}
\end{minipage}

\subsection*{Présentation du système -- Analyse externe}

Le Sys-Reeduc est destiné à aider à la rééducation des membres inférieurs chez les patients ayant été victime d'un accident. Ce système permet une rééducation active, ce qui signifie que l'on cherche à renforcer les muscles et la coordination musculaire. Elle est réalisée en boucle fermée : le patient ne se laisse pas conduire par le système mais résiste au mouvement proposé par la machine.

Les exercices en chaîne fermée permettent au patient de récupérer beaucoup plus rapidement. Le système Sys-Reeduc a l'avantage de proposer des exercices combinant la flexion de la jambe à la rotation du pied de manière à solliciter parfaitement les muscles souhaités. 

Dans le cadre du fonctionnement du système, le kinésithérapeute peut aider à la rééducation des membres inférieurs du patient en agissant sur : 
\begin{itemize}
\item la flexion – extension du genou ;
\item la « vrille » de la cheville (rotation interne-externe).
\end{itemize}

Le système doit aussi permettre la flexion -- extension de la cheville et s’adapter à la morphologie des patients. Enfin, pour des raisons de sécurité, le système ne doit pas blesser le patient. 

Le système doit répondre (entre autres) aux exigences suivantes : 
\begin{center}
\begin{tabular}{p{.35\textwidth}p{.27\textwidth}p{.27\textwidth}}
\hline
\textbf{Exigences} & \textbf{Critères} & \textbf{Niveaux} \\
\hline
\hline
\multirow{4}{.35\textwidth}{Permettre au kinésithérapeute de rééduquer les membres inférieurs du patient}
&
Angle de rotation de la cuisse & De 0\textdegree \; à 150\textdegree \\
& Effort du patient & Jusqu'à 20 N.\\
& Écart de position & Nul \\
& Rapidité & $T_{5\%}<0,2 \; s.$ \\
\hline
\multirow{3}{.35\textwidth}{S'adapter à la morphologie des patients} & Longueur de la cuisse et jambe & De 0,6 à 1,2 m. \\
 & Écartement du bassin & 370 à 600 mm. \\
 & Distance plat du pied -- cheville & \\
\hline
\multirow{2}{.35\textwidth}{Ne pas blesser le patient} & \multirow{2}{.27\textwidth}{Sécurité} & Bloquer le fonctionnement en fonction de la taille du patient \\
\hline
\end{tabular}
\end{center}


\subparagraph{}
\textit{Proposer un diagramme de contexte faisant la liste des entités interagissant avec le système.}

\subparagraph{}
\textit{Proposer un diagramme de cas d'utilisation. Pour cela, on précisera : 
\begin{itemize}
 \item deux acteurs;
 \item un cas d'utilisation principal;
 \item un cas d'utilisation de type <<include>> (cas d'utilisation obligatoirement exécuté);
 \item un cas d'utilisation de type <<extend>> (cas d'utilisation optionnel).
\end{itemize}}

On donne le diagramme des exigences partiel suivant : 

\begin{center}
\includegraphics[width=\textwidth]{images/ExigencesVierges}
\end{center}


\subparagraph{}
\textit{Compléter les exigences \textit{1.2}, \textit{1.2.1} et \textit{1.4.1} du cahier des charges.}

\end{document}


