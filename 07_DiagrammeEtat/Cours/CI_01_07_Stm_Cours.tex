\documentclass[10pt]{article}
\input{style/coursHeadings}
\input{style/programHeadings}
\input{style/macros_SII}
\input{style/macros_Titres}
\input{style/macros_Frames}

%Si le boolen xp est vrai : compilation pour xabi
%Sinon compilation Damien
\newboolean{xp}
\setboolean{xp}{true}

\newboolean{prof}
\setboolean{prof}{true}

\usepackage[%
    pdftitle={CI 01 : Introction à l'IS - Ch 7 : Diagramme d'états},
    pdfauthor={Xavier Pessoles},
    colorlinks=true,
    linkcolor=blue,
    citecolor=magenta]{hyperref}


\def\discipline{Sciences Industrielles de l'Ingénieur}
\def\xxtitre{\ifthenelse{\boolean{xp}}{
CI 1 : Analyse des systèmes pluritechniques et multiphysiques -- Initiation à l'Ingénierie Système}{}}

\def\xxsoustitre{\ifthenelse{\boolean{xp}}{
Chapitre 7 -- Diagramme d'états}{
Partie  -- }}

\def\xxauteur{\ifthenelse{\boolean{xp}}{
Xavier \textsc{Pessoles}}{}}

\def\xxpied{\ifthenelse{\boolean{xp}}{
CI 1 : Introduction à l'IS\\
Ch. 7 : Diagramme d'états -- Cours}{
\xxtitre}}

\def\xxcathegorie{\ifthenelse{\boolean{xp}}{
2013 -- 2014 \\
Xavier \textsc{Pessoles}}{}}





%---------------------------------------------------------------------------


\begin{document}

\ifthenelse{\boolean{xp}}{\input{style/enteteXP}}{\input{style/enteteDI}}





\begin{center}
\includegraphics[width=.95\textwidth]{images/stm} 
\end{center}

\vspace{.2cm}

%
%\begin{prob}
%\textsc{Problématique :}
%\begin{itemize}
%\item 
%\end{itemize}
%\end{prob}

\begin{savoir}
\textsc{Savoirs :}
\begin{itemize}
\item A -- C4.2 : Comportement du système : machine d’état, transition, états, actions, activité, évènement, conditions, état initial, état final.
\end{itemize}
\end{savoir}

\setlength{\parskip}{0ex plus 0.2ex minus 0ex}
 \renewcommand{\contentsname}{}
 \renewcommand{\baselinestretch}{1}

\tableofcontents

 \renewcommand{\baselinestretch}{1.2}
\setlength{\parskip}{2ex plus 0.5ex minus 0.2ex}

% \vspace{1cm}
\textit{Ce document est en évolution permanente. Merci de signaler toutes
erreurs ou coquilles.}

\section{Présentation}
\begin{rem}
Le diagramme d'états (\textit{state machine diagram -- stm}) inclus dans SysML est issus du concept de la machine à états finis (ou automate fini) (\textit{Finite State Machine -- FSM}). 

Un automate est constitué d'une succession d'états et de transition. Le nombre d'états est fini. 
\end{rem}



\begin{defi}
\begin{minipage}[c]{.55\linewidth}
\textbf{État}

Un état représente une situation durant la vie d'un bloc pendant laquelle :
\begin{itemize}
\item il satisfait une certaine condition;
\item il exécute une certaine \textbf{activité};
\item il attend un certain événement. 
\end{itemize}

Un bloc passe par une succession d'état durant son existence. Un état a une durée finie, variable selon la vue du bloc, en particulier en fonction des événements qui lui arrivent. 


\end{minipage} \hfill
\begin{minipage}[c]{.4\linewidth}
\begin{center}
\includegraphics[width=.95\textwidth]{images/etat}
\end{center}
\end{minipage}
\end{defi}





\begin{defi}
\begin{minipage}[c]{.55\linewidth}
\textbf{Transition}

Une transition est un lien entre deux états. On peut y associer un \textbf{événement} (déclencheur), un élément déclencheur appelé \textbf{condition de garde} et un \textbf{effet}. 

Une transition est franchie lorsque l'événement a été réalisé et que la condition de garde est vraie. 
L'effet est produit au passage de la transition. L'état amont est désactivé. Si une activité est en cours (do/...),elle est désactivée. 



Une transition réflexive entraîne une sortie puis une nouvelle entrée dans l'état. 
\end{minipage} \hfill
\begin{minipage}[c]{.4 \linewidth}
\begin{center}
\includegraphics[width=.95\textwidth]{images/Transition}
\end{center}
\end{minipage}
\end{defi}



\begin{defi}
\begin{minipage}[c]{.55\linewidth}
\textbf{Événement}

Il existe quatre types d’évènement associé à une transition :
\begin{itemize}
\item le message (\textit{signal event}) : un message asynchrone est arrivé;
\item l'événement temporel (\textit{time event}) : un intervalle de temps s’est écoulé depuis l’entrée dans un état (mot clé « \textit{after} ») ou un temps absolu a été atteint (mot clé « \textit{at} »);
\item l'événement de changement (\textit{change event}) : une valeur a changé de telle sorte que la transition est franchie (mot clé « \textit{when} »);
\item l'événement d’appel (\textit{call event}) : une requête de fonction (operation) du bloc a été effectuée. Un retour est attendu. Des arguments (paramètres) de fonction peuvent être nécessaires.
\end{itemize}
\end{minipage} \hfill
\begin{minipage}[c]{.4 \linewidth}
\begin{center}
\includegraphics[width=.95\textwidth]{images/evenements}
\end{center}
\end{minipage}
\end{defi}



\begin{defi}
\textbf{Condition de garde}

La condition de garde est une expression booléenne faisant intervenir des entrées et / ou des variables internes. Elle autorise le passage d’un état à un autre. Il est possible d’utiliser les notations non booléennes de front montant ($\uparrow$) et front descendant ($\downarrow$).
\end{defi}



\begin{defi}
\textbf{Effet, action, activité}

L''« effet » associé à une transition peut être une simple action ou une
séquence d’actions. Une action peut représenter la mise à jour d’une valeur, un appel d’opération,
ainsi que l’envoi d’un signal à un autre bloc. L’exécution de l’effet est unitaire et ne permet de traiter
aucun événement supplémentaire pendant son déroulement. Les activités durables (do-activity)
ont une certaine durée, peuvent être interrompues et sont associées aux états.

\end{defi}


\begin{defi}
\begin{minipage}[c]{.7\linewidth}
\textbf{État initial, état final}

Le diagramme d’états comprend deux pseudo-états supplémentaires:
\begin{itemize}
\item l’état initial du diagramme d’états correspond à la création d’une instance ;
\item l’état final du diagramme d’états correspond à la destruction de l’instance.
\end{itemize}

Il est possible d’utiliser plusieurs états finaux pour distinguer par
exemple la destruction normale de l’élément en fin de vie d’une destruction accidentelle.

\end{minipage} \hfill
\begin{minipage}[c]{.25 \linewidth}
\begin{center}
\includegraphics[width=\textwidth]{images/Etat_ini_fin}
\end{center}
\end{minipage}
\end{defi}


\section{Compléments}
\subsection{Les états composites}
\begin{defi}
\textbf{État composite}

Un état composite (aussi appelé super-état) permet d’englober plusieurs sous-états exclusifs. 
Les états composites permettent d'introduire la notion d'état de niveau hiérarchique inférieur et supérieur. 

Les états composites comportent un état initial qui leur est propre.

\begin{center}
\includegraphics[width=\textwidth]{images/EtatComposite}
\end{center}
\end{defi}


\begin{defi}
\textbf{État composite avec régions orthogonales}

Un état composite est dit orthogonal lorsqu'il permet d'obtenir plusieurs états actifs simultanément. Il permet donc de réaliser des activités en parallèle. 
\end{defi}
\subsection{Les pseudos états}

\begin{defi}
\textbf{Pseudo état}

Un pseudo-état est un élément de commande qui influence le comportement d’une machine d’état. Ils peuvent être utilisés dans un diagramme d’états ou dans un diagramme d’activité.

Le formalisme SysML admet neuf pseudo-états :

\begin{tabular}[c]{c p{15cm}}
\includegraphics[width=.5cm]{images/sh}
&
« \textit{shallow history} » : permet à un état de niveau hiérarchique supérieur (état
composite) de se souvenir du dernier sous-état, avant qu’il n’évolue vers un autre état. \\
& \\
\includegraphics[width=.5cm]{images/dh}
&
« \textit{deep history} » : idem que précédemment mais avec la propagation de l’historique à
tous les sous-états composites de niveaux hiérarchiques inférieurs. \\
& \\
\includegraphics[width=.5cm]{images/fork}
&
« \textit{fork} » et « \textit{join} » : divergence et convergence de séquences parallèles. \\
& \\
\includegraphics[width=.5cm]{images/choice}
&
« \textit{choice} » : sélection et convergence de séquences exclusives. Il est nécessaire
qu’une condition avale soit vraie pour que l’évolution du système se poursuive. Les
conditions de gardes doivent être exclusives. Le mot clé « \textit{else} » peut-être utilisé pour
englober tout ce qui n’est pas décrit dans les autres expressions booléennes. Les
conditions de garde avales sont toutes évaluées une fois le pseudo-état atteint. \\
& \\
\includegraphics[width=.5cm]{images/junction}
&
« \textit{junction} » : idem au pseudo-état « \textit{choice} », à la différence que pour qu’un chemin soit emprunté, toutes les conditions de garde, avales et amonts, doivent être vraies.
L’évaluation des conditions avales ne se fait pas une fois le pseudo-état atteint mais
avant.\\
\includegraphics[width=.5cm]{images/entry}
\includegraphics[width=.5cm]{images/exit}
& 
« \textit{entry point} » et « \textit{exit point} » : permet de créer un point d’entrée du diagramme et un point de sortie vers un autre diagramme. \\
& \\
\includegraphics[width=.5cm]{images/terminate}
&
« \textit{terminate} » : permet de terminer une séquence sans destruction de l’instance de bloc.
\end{tabular}
\end{defi}

\begin{center}
\includegraphics[width=.8\textwidth]{images/PseudosEtats}
\end{center}




\begin{thebibliography}{2}
\bibitem{1}{Patrick Beynet et Al. , \textit{Sciences Industrielles de l'Ingénieur, PCSI -- MPSI}, Éditions Ellipses.}
\bibitem{2}{Pascal Roques, \textit{SysML par l'exemple}, Éditions Eyrolles.}
\bibitem{3}{Pierre Debout, \textit{Automatique -- Systèmes à événements discrets}, Cours de PCSI.}
\end{thebibliography}
\end{document}