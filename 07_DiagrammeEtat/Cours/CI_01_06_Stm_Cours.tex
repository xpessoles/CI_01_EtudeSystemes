\documentclass[10pt]{article}
\input{style/coursHeadings}
\input{style/programHeadings}
\input{style/macros_SII}
\input{style/macros_Titres}
\input{style/macros_Frames}

%Si le boolen xp est vrai : compilation pour xabi
%Sinon compilation Damien
\newboolean{xp}
\setboolean{xp}{true}

\newboolean{prof}
\setboolean{prof}{true}

\usepackage[%
    pdftitle={CI 01 : Introction à l'IS - Ch 7 : Diagramme d'états},
    pdfauthor={Xavier Pessoles},
    colorlinks=true,
    linkcolor=blue,
    citecolor=magenta]{hyperref}


\def\discipline{Sciences Industrielles de l'Ingénieur}
\def\xxtitre{\ifthenelse{\boolean{xp}}{
CI 1 : Analyse des systèmes pluritechniques et multiphysiques -- Initiation à l'Ingénierie Système}{}}

\def\xxsoustitre{\ifthenelse{\boolean{xp}}{
Chapitre 7 -- Diagramme d'états}{
Partie  -- }}

\def\xxauteur{\ifthenelse{\boolean{xp}}{
Xavier \textsc{Pessoles}}{}}

\def\xxpied{\ifthenelse{\boolean{xp}}{
CI 1 : Introduction à l'IS\\
Ch. 7 : Diagramme d'états -- Cours}{
\xxtitre}}

\def\xxcathegorie{\ifthenelse{\boolean{xp}}{
2013 -- 2014 \\
Xavier \textsc{Pessoles}}{}}





%---------------------------------------------------------------------------


\begin{document}

\ifthenelse{\boolean{xp}}{\input{style/enteteXP}}{\input{style/enteteDI}}





\begin{center}
\includegraphics[width=.95\textwidth]{images/stm} 
\end{center}

\vspace{.2cm}

%
%\begin{prob}
%\textsc{Problématique :}
%\begin{itemize}
%\item 
%\end{itemize}
%\end{prob}

\begin{savoir}
\textsc{Savoirs :}
\begin{itemize}
\item %A-C2.3 : Diagramme de séquence de SysML.
\item %A-C2-S2 : Identifier les interactions entre les acteurs et le système étudié. 
\end{itemize}
\end{savoir}

\setlength{\parskip}{0ex plus 0.2ex minus 0ex}
 \renewcommand{\contentsname}{}
 \renewcommand{\baselinestretch}{1}

\tableofcontents

 \renewcommand{\baselinestretch}{1.2}
\setlength{\parskip}{2ex plus 0.5ex minus 0.2ex}

% \vspace{1cm}
\textit{Ce document est en évolution permanente. Merci de signaler toutes
erreurs ou coquilles.}

\section{Présentation}
\begin{rem}
Le diagramme d'états (\textit{state machine diagram -- stm}) inclus dans SysML est issus du concept de la machine à états finis (ou automate fini) (\textit{Finite State Machine -- FSM}). 

Un automate est constitué d'une succession d'états et de transition. Le nombre d'états est fini. 
\end{rem}



\begin{defi}
\begin{minipage}[c]{.55\linewidth}
\textbf{État}

Un état représente une situation durant la vie d'un bloc pendant laquelle :
\begin{itemize}
\item il satisfait une certaine condition;
\item il exécute une certaine \textbf{activité};
\item il attend un certain événement. 
\end{itemize}

Un bloc passe par une succession d'état durant son existence. Un état a une durée finie, variable selon la vue du bloc, en particulier en fonction des événements qui lui arrivent. 


\end{minipage} \hfill
\begin{minipage}[c]{.4\linewidth}
\begin{center}
\includegraphics[width=.95\textwidth]{images/etat}
\end{center}
\end{minipage}
\end{defi}





\begin{defi}
\begin{minipage}[c]{.55\linewidth}
\textbf{Transition -- Condition de garde}

Une transition est un lien entre bloc. Il peut contenir un élément déclencheur appelé \textbf{condition de garde}. Cette condition est une expression booléenne. Elle est notée entre crochets.

Pour que la transition soit franchie, il faut que la condition de garde soit vraie. 

Le déclencheur (ou événement déclencheur) peut par exemple être aussi un événement temporel, un signal...

 Les transitions peuvent être réflexives. 

\end{minipage} \hfill
\begin{minipage}[c]{.4 \linewidth}
\begin{center}
\includegraphics[width=.95\textwidth]{images/Transition}
\end{center}
\end{minipage}
\end{defi}


\begin{defi}
\begin{minipage}[c]{.8\linewidth}
\textbf{Effet, action, activité}

Une transition peut spécifier un comportement optionnel réalisé par le bloc lorsque la transition est
déclenchée. Ce comportement est appelé « effet » : cela peut être une simple action ou une
séquence d’actions. Une action peut représenter la mise à jour d’une valeur, un appel d’opération,
ainsi que l’envoi d’un signal à un autre bloc. L’exécution de l’effet est unitaire et ne permet de traiter
aucun événement supplémentaire pendant son déroulement. Les activités durables (do-activity)
ont une certaine durée, peuvent être interrompues et sont associées aux états.

\end{minipage} \hfill
\begin{minipage}[c]{.15 \linewidth}
\begin{center}
%\includegraphics[width=.95\textwidth]{images/LigneVie}
\end{center}
\end{minipage}
\end{defi}


\begin{defi}
\begin{minipage}[c]{.8\linewidth}
\textbf{État initial, état final}

En plus de la succession d’états « normaux » correspondant au cycle de vie d’un bloc, le diagramme
d’états comprend également deux pseudo-états :
\begin{itemize}
\item l’état initial du diagramme d’états correspond à la création d’une instance ;
\item l’état final du diagramme d’états correspond à la destruction de l’instance;
\item il est possible d’utiliser plusieurs états finals (ou finaux, les deux se disent...) pour distinguer par
exemple la destruction normale de l’élément en fin de vie d’une destruction accidentelle.
\end{itemize}

\end{minipage} \hfill
\begin{minipage}[c]{.15 \linewidth}
\begin{center}
%\includegraphics[width=.95\textwidth]{images/LigneVie}
\end{center}
\end{minipage}
\end{defi}


\begin{defi}
\begin{minipage}[c]{.8\linewidth}
\textbf{État}

\end{minipage} \hfill
\begin{minipage}[c]{.15 \linewidth}
\begin{center}
%\includegraphics[width=.95\textwidth]{images/LigneVie}
\end{center}
\end{minipage}
\end{defi}


\begin{thebibliography}{2}
\bibitem{1}{Patrick Beynet et Al. , \textit{Sciences Industrielles de l'Ingénieur, PCSI -- MPSI}, Éditions Ellipses.}
\bibitem{2}{Pascal Roques, \textit{SysML par l'exemple}, Éditions Eyrolles.}
 
\end{thebibliography}
\end{document}