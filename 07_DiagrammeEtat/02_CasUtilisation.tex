\documentclass[11pt,oneside]{article}
\input{coursHeadings}
\usepackage[raccourcis]{FAST}
\usepackage[%
    pdftitle={IS -- SysML -- Cas d'utilisation},
    pdfauthor={Xavier Pessoles},
    colorlinks=true,
    linkcolor=blue,
    citecolor=magenta]{hyperref}

\usepackage{pifont}


% \makeatletter \let\ps@plain\ps@empty \makeatother
%% DEBUT DU DOCUMENT
%% =================
\sloppy
\hyphenpenalty 10000

\newcommand{\Pointilles}[1][3]{%
\multido{}{#1}{\makebox[\linewidth]{\dotfill}\\[\parskip]
}}


\colorlet{shadecolor}{orange!15}

\newtheorem{theorem}{Theorem}


\begin{document}


\newboolean{prof}
\setboolean{prof}{true}
%------------- En tetes et Pieds de Pages ------------
\pagestyle{fancy}
\renewcommand{\headrulewidth}{0pt}

\fancyhead{}
\fancyhead[L]{%
\noindent\noindent\begin{minipage}[c]{2.6cm}
%Lycée Rouvière PTSI
\includegraphics[width=2cm]{png/logo_ptsi.png}%
\end{minipage}
}


\fancyhead[C]{\rule{12cm}{.5pt}}

\fancyhead[R]{%
\noindent\begin{minipage}[c]{3cm}
\begin{flushright}
\footnotesize{\textit{\textsf{Sciences Industrielles\\ de l'Ingénieur}}}%
\end{flushright}
\end{minipage}
}

\renewcommand{\footrulewidth}{0.2pt}

\fancyfoot[C]{\footnotesize{\bfseries \thepage}}
\fancyfoot[L]{\footnotesize{2012 -- 2013} \\ X. \textsc{Pessoles}}
\ifthenelse{\boolean{prof}}{%
\fancyfoot[R]{\footnotesize{Cours -- CI 1.02 : IS -- SysML -- UC -- P}}
}{%
\fancyfoot[R]{\footnotesize{Cours -- CI 6 : PPM}}
}



\begin{center}
 \huge\textsc{CI 1 -- IS}

 \large\textsc{Étude des systèmes pluritechniques et multiphysiques - Initiation à l'Ingénierie Système}
\end{center}

\begin{center}
 \LARGE\textsc{Chapitre 2 -- SysML -- Diagramme des cas d'utilisation}
\end{center}

%\begin{flushright}
%\textit{D'après documents de Jean-Pierre Pupier}
%\end{flushright}

\vspace{.5cm}

A-C2.1	Frontière d'étude, fonction globale et performance, cas d’utilisation, acteurs (humain ou systèmes connectés), interactions fonctionnelles, relations entre cas d’utilisation	
A-C2.2	Diagramme des cas d’utilisation de sysML	

Pour montrer les interactions entre les acteurs et le système étudié, le diagramme des cas d’utilisation est particulièrement adapté. 

%\begin{minipage}[c]{.23\linewidth}
%\begin{center}
%\includegraphics[height=2.5cm]{png/tour_bois}
%
%\textit{Tour à bois \cite{tab}}
%\end{center}
%\end{minipage} \hfill
%\begin{minipage}[c]{.23\linewidth}
%\begin{center}
%\includegraphics[height=2.5cm]{png/cazeneuve}
%
%\textit{Tour conventionnel \cite{cazeneuve}}
%\end{center}
%\end{minipage} \hfill
%\begin{minipage}[c]{.23\linewidth}
%\begin{center}
%\includegraphics[height=2.5cm]{png/tour_mazak}
%
%\textit{Tour à commande numérique \cite{mazak}}
%\end{center}
%\end{minipage}\hfill
%\begin{minipage}[c]{.23\linewidth}
%\begin{center}
%\includegraphics[height=2.5cm]{png/plaquettes}
%
%\textit{Plaquettes de tournage \cite{plaquettes}}
%\end{center}
%\end{minipage}

\vspace{.5cm}

\begin{center}
%\includegraphics[height=7cm]{png/cycleV}
\end{center}

%\begin{center}
%\includegraphics[width=.9\textwidth]{png/cyclev.png}

%\textit{Cycle de conception d'un produit}
%\end{center}

%\begin{prob}
%\textsc{Problématique :}
%\begin{itemize}
%\item %Quelles sont les conditions fonctionnelles permettant le fonctionnement du système ?
%\item %Quelle est la chaîne de côte unidirectionnelle correspondant à une condition donnée ?
%\end{itemize}
%\end{prob}



\begin{savoir}
\textsc{Savoirs :}
%\begin{itemize}
%\end{itemize}
\end{savoir}
 

\setlength{\parskip}{0ex plus 0.2ex minus 0ex}
 \renewcommand{\contentsname}{}
 \renewcommand{\baselinestretch}{1}

\tableofcontents

 \renewcommand{\baselinestretch}{1.2}
\setlength{\parskip}{2ex plus 0.5ex minus 0.2ex}

% \vspace{1cm}
\textit{Ce document évolue. Merci de signaler toutes erreurs ou coquilles.}

\section{Présentation}
\begin{defi}

\end{defi}

Permet d'avoir un point de vue utilisateur / acteur
acteur = humain ou autre système interacteur

fonctionnalités visibles de l'extérieur (mais pas toutes les fonctionnalités)


Acteur = bonhomme batonbaton

cadre = frontière du système



Acteur : 
un humain ayant un rôle qui va interagir avec le système
un système
tout ce qui est en INTERaction avec le système

acteur primaire plutot a gauche
acteur secondaire plutot a droite

interaction = "communication" et pas d'interaction purement physique


Frontière du système : représenté par un cadre
le cadre symbolise la frontièer du système

cette frontière doit être bien comprise et définie. 
elle englobe ce qui n'est pas présent avant l'installation du système.
Le reste représente souvent un acteur



Cas d'utilisation dans un ovale

Il est énoncé par un acteur PRIMAIRE

L'acteur dit " moi je veux"
Il énonce le résultat qu'il attend
Mission que l'on cherche à atteindre à travers le système

Cas d'utilisation possède un élément déclencheur
Succession d'étape
Le service est rendu

Mettre des éléments qui interagissent réellement avec le système (pas le soleil, le vent ...)




(Attention : maintenance = exigence)


Descritpion des scénaros
decrit le scenario nominal
les scenarios alternarifs
les scénarios d'echecs
mais aussi les préconditions (l "état du systeme pour que la cas d'utilisation puisse démarrer)
les postconditions (ce qui a changé dans l'état du système a la fin du cas d'utilisation)


Attention
ne pas descendre trop bas en terme de détail (cibler les fonctionnalités premieres)
le cas d'utilisation ne doit donc pas se réduire systèmatiquement a une seule sequence
encore moins a une simple action


Pour aller plus loin
relation entre cas d'utilisation
inclusion - stereotype include
extension - extend
généralisation / spécialisation (fleche blanche) pas de stereotype

\section{Exemple}

\begin{thebibliography}{2}
\bibitem{tab}{\url{http://bois.fordaq.com/fordaq/srvAuctionView.html?AucTIid=17877844}}

\end{thebibliography}

\end{document}