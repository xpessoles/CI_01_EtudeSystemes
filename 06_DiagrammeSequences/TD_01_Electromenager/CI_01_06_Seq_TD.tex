\documentclass[10pt]{article}
\input{style/coursHeadings}
\input{style/programHeadings}
\input{style/macros_SII}
\input{style/macros_Titres}
\input{style/macros_Frames}

%Si le boolen xp est vrai : compilation pour xabi
%Sinon compilation Damien
\newboolean{xp}
\setboolean{xp}{true}

\newboolean{prof}
\setboolean{prof}{false}

\usepackage[%
    pdftitle={CI 01 : Introction à l'IS - Ch 6 : Diagramme de Séquence},
    pdfauthor={Xavier Pessoles},
    colorlinks=true,
    linkcolor=blue,
    citecolor=magenta]{hyperref}


\def\discipline{Sciences Industrielles de l'Ingénieur}
\def\xxtitre{\ifthenelse{\boolean{xp}}{
CI 1 : Analyse des systèmes pluritechniques et multiphysiques -- Initiation à l'Ingénierie Système}{}}

\def\xxsoustitre{\ifthenelse{\boolean{xp}}{
Chapitre 6 -- Diagramme de Séquence}{
Partie  -- }}

\def\xxauteur{\ifthenelse{\boolean{xp}}{
Xavier \textsc{Pessoles}}{}}

\def\xxpied{\ifthenelse{\boolean{xp}}{
CI 1 : Introduction à l'IS\\
Ch. 6 : Diagramme de Séquence -- Cours}{
\xxtitre}}

\def\xxcathegorie{\ifthenelse{\boolean{xp}}{
2013 -- 2014 \\
Xavier \textsc{Pessoles}}{}}





%---------------------------------------------------------------------------


\begin{document}

\ifthenelse{\boolean{xp}}{\input{style/enteteXP}}{\input{style/enteteDI}}

\begin{flushright}
\textit{D'après ressources de David Prévost.}
\end{flushright}


\section*{Bouilloire électrique}

\begin{minipage}[c]{.65\linewidth}
On s’intéresse à une bouilloire électrique. Une bouilloire est généralement composée d’un pot pour recevoir l’eau à chauffer ou à faire bouillir, et d’un socle, sur lequel on pose la bouilloire pour faire chauffer l’eau.

L’eau est chauffée grâce à une résistance électrique placée au fond du pot. Un cordon électrique lié au socle permet de brancher la bouilloire et la mise sous tension est réalisée grâce au bouton Marche/Arrêt. Un voyant précise à l’utilisateur si le système est sous tension. Un capteur de température détecte lorsque l’eau bout et coupe la mise sous tension en déclenchant le bouton Marche/Arrêt. L’utilisateur peut interrompre le chauffage à tout moment en mettant le bouton Marche/Arrêt sur « Arrêt ».

Enfin, un couvercle permet de verser l’eau à chauffer, tandis qu’un bec et un filtre permettent de filtrer l’eau chaude que l’on souhaite utiliser.

\end{minipage}\hfill
\begin{minipage}[c]{.3\linewidth}
\begin{center}
\includegraphics[width=.7\textwidth]{images/bouilloire_01}
\includegraphics[width=\textwidth]{images/bouilloire_02}
\end{center}
\end{minipage}

On souhaite maintenant décrire le fonctionnement du système de manière dynamique. Pour cela, on utilise la représentation par diagrammes de séquences.

\subparagraph{}
\textit{Proposer un diagramme de séquence simple illustrant le fonctionnement d’une bouilloire électrique (On utilisera deux lignes de vie : une matérialisant l’utilisateur, la deuxième la bouilloire).}
\begin{center}
\includegraphics[width=.7\textwidth]{images/bouilloire_03}
\end{center}


\subparagraph{}
\textit{Proposer une première machine à états ne décrivant que le fonctionnement automatique de la bouilloire (ne pas tenir compte de l’arrêt manuel).}


\subparagraph{}
\textit{Compléter le diagramme proposé à la question précédente, en incluant la possibilité d’un arrêt manuel.}


\ifthenelse{\boolean{prof}}{
\begin{center}
\includegraphics[width=.7\textwidth]{images/bouilloire_04}
\includegraphics[width=.7\textwidth]{images/bouilloire_05}
\includegraphics[width=.7\textwidth]{images/bouilloire_06}
\end{center}
}{}

\section*{Store automatisé}
\begin{minipage}[c]{.3\linewidth}
\begin{center}
\includegraphics[width=\textwidth]{images/store_01}
\end{center}
\end{minipage}\hfill
\begin{minipage}[c]{.67\linewidth}
\begin{center}
\includegraphics[width=\textwidth]{images/store_02}
\end{center}
\end{minipage}

Ce store est motorisé et réagit automatiquement aux conditions météorologiques. Il se déroule automatiquement dès que l'intensité lumineuse atteint un niveau élevé. Il s'enroule dès que le vent se lève ou dès que la luminosité retourne à un bas niveau. L'utilisateur peut également intervenir sur son fonctionnement. Néanmoins la présence de vent est analysée en priorité pour une raison de sécurité. 

On souhaite décrire le fonctionnement du système de manière dynamique. Pour cela, on utilise la représentation par diagrammes de séquences.

Voici un diagramme de séquence du store étudié.

\begin{center}
\includegraphics[width=\textwidth]{images/store_03}
\end{center}

\setcounter{subparagraph}{0}
\subparagraph{}
\textit{Quelle est la séquence (cycle d’utilisation) décrite par la Figure 4 ci-dessus ?}

\subparagraph{}
\textit{Expliquer le fonctionnement de la boucle (loop).}

\subparagraph{}
\textit{Réaliser le diagramme de séquence de l’autre cycle d’utilisation.}
\subparagraph{}
\textit{Proposer un diagramme d’état (machine à états) décrivant le fonctionnement du store.}


\ifthenelse{\boolean{prof}}{
\begin{center}
\includegraphics[width=.7\textwidth]{images/store_04}
\includegraphics[width=.7\textwidth]{images/store_05}
\end{center}
}{}
\end{document}